% !TeX spellcheck = en_US

\documentclass[a4paper]{article}
\usepackage[utf8x]{inputenc}	% Use UTF8 characters
\usepackage[hyphens]{url}	% Proper url in the references section
\usepackage{relsize}		% Provide \mathlarger to get formula's to the correct size
\usepackage{hyperref}		% Clickable urls and references ("\pdfendlink ended up in different nesting level than \pdfstartlink"? add [draft])
\usepackage{enumitem}		% Numbered items, enumerate customizers
\usepackage{graphicx}		% Images
\usepackage{tabularx}		% Tables
\usepackage{needspace}		% Prevent sections starting when low on space
\usepackage{placeins}		% \FloatBarrier
\usepackage{longtable}		% Multi-page tables
\usepackage[nottoc,numbib]{tocbibind} % Add reference to the table of contents (and give it a chapter number)

\emergencystretch=10pt		% Allow more stretching whitespace to prevent overfull/underfilled lines.

% TODO: use \Needspace{5\baselineskip} before sections (redefine sections?)
\usepackage[usenames, dvipsnames]{color}
\usepackage[toc,page]{appendix}
\definecolor{inline}{rgb}{0,0,0.5}
\pagenumbering{arabic}		% Page numbering
\setlist{itemsep=0pt, topsep=2pt} % Smaller spacing between items
% \overlap for week numbers that are used twice
\definecolor{overlap}{gray}{0.5}
\newcommand{\overlap}[1]{\textcolor{overlap}{#1}}

\newcommand{\sectionbreak}{\clearpage} % Newpage before starting new section

% Setup section header spacing
\usepackage{titlesec}
\titlespacing*{\section}
	{0pt}{0.7cm}{0cm}
\titlespacing*{\subsection}
	{0pt}{0.7cm minus 0.4cm}{0cm}
\titlespacing*{\subsubsection}
	{0pt}{0.4cm minus 0.2cm}{0cm}

% Setup captions styling
\usepackage{caption}
\captionsetup[figure]{format=plain, singlelinecheck=false, margin=0pt, font={bf,footnotesize}, justification=centering, aboveskip=3pt}
\captionsetup[table]{format=plain, singlelinecheck=false, margin=0pt, font={bf,footnotesize}, justification=centering, aboveskip=3pt}
\captionsetup[lstlisting]{format=plain, singlelinecheck=false, margin=0pt, font={bf,footnotesize}, justification=centering, aboveskip=3pt}

% Setup todo note command
\marginparsep 11pt
\marginparwidth 1.4cm
\usepackage{setspace}	% \setstretch
\newcommand{\todo}[1]{{\color{BurntOrange}\sffamily\textbf{todo: #1}\par}}

\setlength{\parindent}{0em} % No indenting for new paragraphs
\setlength{\parskip}{0.5em} % Empty line between paragraphs

% Spacing of tableofcontents
\usepackage{tocloft}
\renewcommand\cftsecafterpnum{\vskip0pt}

% Code listings
%===== Code snippet styling =====
\newcommand{\code}[1]{\texttt{\small \color{inline}#1}} % \code command for inline snippets
\usepackage{listings}		% Code snippets
\usepackage{color}			% Code highlighting colors
\definecolor{dkgreen}{rgb}{0,0.6,0}
\definecolor{inline}{rgb}{0,0,0.5}
\definecolor{gray}{gray}{0.2}
\definecolor{codeText}{rgb}{0.1,0.1,0.1}
\definecolor{mauve}{rgb}{0.58,0,0.82}
\lstset{frame=tb,
	framerule=0.2pt,
	language=Java,
	aboveskip=2mm,
	belowskip=5mm,
	showstringspaces=false,
	columns=flexible,
	basicstyle={\small\ttfamily\color{codeText}},
	numbers=left,
	numbersep=4pt,
	numberstyle=\tiny\color{gray},
	keywordstyle=\color{blue},
	commentstyle=\color{dkgreen},
	stringstyle=\color{mauve},
	breaklines=true,
	breakatwhitespace=true,
	tabsize=4
}
\lstdefinelanguage{YAML}
{%
	alsoother     = @\$,
	morecomment   = **[l]{\#},
	morecomment   = **[s]{/*}{*/},
	morestring    = **[s]{"}{"},
}[keywords,strings,comments]
\lstset{language=YAML} % Set default language

\lstdefinelanguage{JSON}
{%
	string       = [s]{"}{"},
	stringstyle  = \color{blue},
	comment      = [l]{:},
	commentstyle = \color{black},
}[strings,comments]

\begin{document}

\begin{titlepage}
	\begin{center}
		{\huge\bfseries Verification of the Prefix Sum Program in an OpenCL Environment\par}
		
		\vspace{1cm}
		{\LARGE Thijs Wiefferink\par}
		{\large thijs@wiefferink.me, s1366564}
		
		\vfill
		
		{\Large
			University of Twente		
		}
	\end{center}

\end{titlepage}
\newpage


\section*{Abstract}
\todo{write}


\section*{Keywords}
\todo{write}


\section*{Preface}
This project is the continuation of my bachelor thesis project, in which the verification of the prefix sum algorithm has been started. Since only the 'data race free' part has been proven for the first part of the algorithm (the upsweep), the goal of this project is to prove the complete algorithm data race free, and additionally prove the functionality of the algorithm.

The necessary background information will be included in this report to understand the goal and results of the project, but full details of the Bachelor project can be read in the paper of that project \cite{bachelorThesis}.


\newpage
\tableofcontents


\section{Introduction}
This chapter describes the research domain, shows the problem that is solved, introduces the research questions and explains the approach.

\subsection{Research domain}
A graphics processing unit (GPU) is a device designed to rapidly manipulate and alter memory to accelerate the creation of images, for example while watching a video or playing a game. However, GPUs are also used more for general purpose computing, which is traditionally handled by the central processing unit (CPU). GPUs are better than CPUs doing parallel execution on large data sets. For example increasing the brightness of an image is easily done by a GPU, since this operation can be done in parallel on all pixels of the image. GPUs are however also used for physics calculations or mining crypto currencies.

\todo{verification language, VerCors}

\subsection{Problem statement}
\todo{use of verification}

\subsection{Research question}
\todo{write}

\subsection{Approach}
\todo{verification process and problem solving}

\subsection{Report structure}
\todo{introduce chapters}


\section{Prefix sum algorithm}
\todo{explain parallel prefix sum, based on bachelorreferaat paper}


\section{Verifying permissions}
\todo{verification process of the read/write permissions of the array (initial based on bachelorreferaat, extended to allow functional verification)}


\section{Verifying functionality}
\todo{verification process of the functionality of the program}


\section{Discussion}
\todo{summary}

\subsection{Results}
\todo{result description: verified}

\subsection{Related work}
\todo{similar verification projects}


\section{Conclusion}
\todo{summary}

\subsection{Research question answer}
\todo{works, not for huge programs though}

\subsection{Limitations and problems}
\todo{tool limitations, change code for verification process}

\subsection{Research value}
\todo{do larger programs, improve tool, concurrency}

\subsection{Future work}
\todo{larger programs, improve tool}


\bibliographystyle{abbrv}
\bibliography{Report}

\end{document}
